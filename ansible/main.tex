\documentclass[10pt,a4paper]{article}

% Packages
\usepackage{fancyhdr}           % For header and footer
\usepackage{multicol}           % Allows multicols in tables
\usepackage{tabularx}           % Intelligent column widths
\usepackage{tabulary}           % Used in header and footer
\usepackage{hhline}             % Border under tables
\usepackage{graphicx}           % For images
\usepackage{xcolor}             % For hex colours
%\usepackage[utf8x]{inputenc}    % For unicode character support
\usepackage[T1]{fontenc}        % Without this we get weird character replacements
\usepackage{colortbl}           % For coloured tables
\usepackage{setspace}           % For line height
\usepackage{lastpage}           % Needed for total page number
\usepackage{seqsplit}           % Splits long words.
%\usepackage{opensans}          % Can't make this work so far. Shame. Would be lovely.
\usepackage[normalem]{ulem}     % For underlining links
% Most of the following are not required for the majority
% of cheat sheets but are needed for some symbol support.
\usepackage{amsmath}            % Symbols
\usepackage{MnSymbol}           % Symbols
\usepackage{wasysym}            % Symbols
%\usepackage[english,german,french,spanish,italian]{babel}              % Languages

% Lengths and widths
\addtolength{\textwidth}{6cm}
\addtolength{\textheight}{-1cm}
\addtolength{\hoffset}{-3cm}
\addtolength{\voffset}{-2cm}
\setlength{\tabcolsep}{0.2cm} % Space between columns
\setlength{\headsep}{-12pt} % Reduce space between header and content
\setlength{\headheight}{85pt} % If less, LaTeX automatically increases it
\renewcommand{\footrulewidth}{0pt} % Remove footer line
\renewcommand{\headrulewidth}{0pt} % Remove header line
\renewcommand{\seqinsert}{\ifmmode\allowbreak\else\-\fi} % Hyphens in seqsplit
% This two commands together give roughly
% the right line height in the tables
\renewcommand{\arraystretch}{1.3}
\onehalfspacing

% Commands
\newcommand{\SetRowColor}[1]{\noalign{\gdef\RowColorName{#1}}\rowcolor{\RowColorName}} % Shortcut for row colour
\newcommand{\mymulticolumn}[3]{\multicolumn{#1}{>{\columncolor{\RowColorName}}#2}{#3}} % For coloured multi-cols
\newcolumntype{x}[1]{>{\raggedright}p{#1}} % New column types for ragged-right paragraph columns
\newcommand{\tn}{\tabularnewline} % Required as custom column type in use

% Font and Colours
\definecolor{HeadBackground}{HTML}{333333}
\definecolor{FootBackground}{HTML}{666666}
\definecolor{TextColor}{HTML}{333333}
\definecolor{DarkBackground}{HTML}{A32222}
\definecolor{LightBackground}{HTML}{F9F1F1}
\renewcommand{\familydefault}{\sfdefault}
\color{TextColor}

% Código para exportar el título y autor personalizado
% Requiere el paquete TikZ
\usepackage{tikz}

% Configuración del título personalizado
\makeatletter
\renewcommand*{\maketitle}{%
\noindent
\begin{minipage}{0.4\textwidth}
\begin{tikzpicture}
\node[rectangle,rounded corners=6pt,inner sep=10pt,fill=red!50!black,text width= 0.95\textwidth] {\color{white}\Huge \@title};
\end{tikzpicture}
\end{minipage}
\hfill
\begin{minipage}{0.55\textwidth}
\begin{tikzpicture}
\node[rectangle,rounded corners=3pt,inner sep=10pt,draw=red!50!black,text width= 0.95\textwidth] {\LARGE \@author};
\end{tikzpicture}
\end{minipage}
}%
\makeatother

% Definición del título, autor y fecha
\title{Ansible}
\author{Pablo de la Cuesta García}
\date{2024}


\begin{document}

% Llamamos al título
\maketitle

\raggedright
\raggedcolumns

% Set font size to small. Switch to any value
% from this page to resize cheat sheet text:
% www.emerson.emory.edu/services/latex/latex_169.html
\footnotesize % Small font.

\begin{multicols*}{2}

\begin{tabularx}{8.4cm}{x{2.96 cm} x{5.04 cm} }
\SetRowColor{DarkBackground}
\mymulticolumn{2}{x{8.4cm}}{\bf\textcolor{white}{ansible (ad-hoc command)}}  \tn
% Row 0
\SetRowColor{LightBackground}
Syntax & ansible \textless{}hosts\textgreater{} -m \textless{}module\textgreater{} -a "\textless{}module arguments\textgreater{}" \tn 
% Row Count 3 (+ 3)
% Row 1
\SetRowColor{white}
Ping all servers & ansible all -m ping \tn 
% Row Count 5 (+ 2)
% Row 2
\SetRowColor{LightBackground}
Execute a shell command & ansible webservers -m shell -a "uptime" \tn 
% Row Count 7 (+ 2)
% Row 3
\SetRowColor{white}
Check disk space & ansible all -m shell -a "df -h" \tn 
% Row Count 9 (+ 2)
\hhline{>{\arrayrulecolor{DarkBackground}}--}
\end{tabularx}
\par\addvspace{1.3em}

\begin{tabularx}{8.4cm}{x{4.08 cm} x{3.92 cm} }
\SetRowColor{DarkBackground}
\mymulticolumn{2}{x{8.4cm}}{\bf\textcolor{white}{ansible-inventory}}  \tn
% Row 0
\SetRowColor{LightBackground}
View entire inventory in JSON format & ansible-inventory -{}-list \tn 
% Row Count 2 (+ 2)
% Row 1
\SetRowColor{white}
View information for a specific group & ansible-inventory -{}-graph webservers \tn 
% Row Count 4 (+ 2)
% Row 2
\SetRowColor{LightBackground}
Export inventory to YAML & ansible-inventory -{}-list -y \tn 
% Row Count 6 (+ 2)
\hhline{>{\arrayrulecolor{DarkBackground}}--}
\SetRowColor{LightBackground}
\mymulticolumn{2}{x{8.4cm}}{El inventario define los hosts y grupos sobre los que Ansible ejecutará las tareas. \newline Características del Inventario: \newline   - Puede ser estático (archivo) o dinámico (script) \newline   - Soporta grupos y subgrupos \newline   - Permite definir variables por host y por grupo \newline   - Puede tener múltiples inventarios (producción, staging, desarrollo)}  \tn 
\hhline{>{\arrayrulecolor{DarkBackground}}--}
\end{tabularx}
\par\addvspace{1.3em}

\begin{tabularx}{8.4cm}{x{3.84 cm} x{4.16 cm} }
\SetRowColor{DarkBackground}
\mymulticolumn{2}{x{8.4cm}}{\bf\textcolor{white}{ansible-vault}}  \tn
% Row 0
\SetRowColor{LightBackground}
Create new encrypted file & ansible-vault create secrets.yml \tn 
% Row Count 2 (+ 2)
% Row 1
\SetRowColor{white}
Encrypt existing file & ansible-vault encrypt vars.yml \tn 
% Row Count 4 (+ 2)
% Row 2
\SetRowColor{LightBackground}
Edit encrypted file & ansible-vault edit secrets.yml \tn 
% Row Count 6 (+ 2)
% Row 3
\SetRowColor{white}
Decrypt file & ansible-vault decrypt secrets.yml \tn 
% Row Count 8 (+ 2)
% Row 4
\SetRowColor{LightBackground}
View content without decrypting & ansible-vault view secrets.yml \tn 
% Row Count 10 (+ 2)
\hhline{>{\arrayrulecolor{DarkBackground}}--}
\SetRowColor{LightBackground}
\mymulticolumn{2}{x{8.4cm}}{Encrypts and decrypts sensitive files.}  \tn 
\hhline{>{\arrayrulecolor{DarkBackground}}--}
\end{tabularx}
\par\addvspace{1.3em}

\begin{tabularx}{8.4cm}{x{4.16 cm} x{3.84 cm} }
\SetRowColor{DarkBackground}
\mymulticolumn{2}{x{8.4cm}}{\bf\textcolor{white}{ansible-console}}  \tn
% Row 0
\SetRowColor{LightBackground}
Start interactive console & ansible-console \tn 
% Row Count 2 (+ 2)
% Row 1
\SetRowColor{white}
Inside the console & webservers.ping \tn 
% Row Count 3 (+ 1)
% Row 2
\SetRowColor{LightBackground}
Inside the console & webservers.shell uptime \tn 
% Row Count 5 (+ 2)
\hhline{>{\arrayrulecolor{DarkBackground}}--}
\end{tabularx}
\par\addvspace{1.3em}

\begin{tabularx}{8.4cm}{x{3.52 cm} x{4.48 cm} }
\SetRowColor{DarkBackground}
\mymulticolumn{2}{x{8.4cm}}{\bf\textcolor{white}{ansible-playbook}}  \tn
% Row 0
\SetRowColor{LightBackground}
Basic syntax & ansible-playbook {[}options{]} playbook.yml \tn 
% Row Count 2 (+ 2)
% Row 1
\SetRowColor{white}
Run a playbook & ansible-playbook site.yml \tn 
% Row Count 4 (+ 2)
% Row 2
\SetRowColor{LightBackground}
Run in check mode (simulation) & ansible-playbook site.yml -{}-check \tn 
% Row Count 6 (+ 2)
% Row 3
\SetRowColor{white}
Run with extra variables & ansible-playbook site.yml -{}-extra-vars "variable=value" \tn 
% Row Count 9 (+ 3)
% Row 4
\SetRowColor{LightBackground}
\# Run limiting to certain hosts & ansible-playbook site.yml -{}-limit webservers \tn 
% Row Count 11 (+ 2)
\hhline{>{\arrayrulecolor{DarkBackground}}--}
\SetRowColor{LightBackground}
\mymulticolumn{2}{x{8.4cm}}{Los playbooks son archivos YAML que definen la configuración deseada y las tareas a ejecutar. \newline Características de los Playbooks: \newline   - Definen una serie de plays (jugadas) \newline   - Cada play asocia hosts con roles y tareas \newline   - Pueden incluir variables, handlers y roles \newline   - Soportan condicionales y bucles \newline   - Permiten reutilización de código}  \tn 
\hhline{>{\arrayrulecolor{DarkBackground}}--}
\end{tabularx}
\par\addvspace{1.3em}

\begin{tabularx}{8.4cm}{x{5.44 cm} x{2.56 cm} }
\SetRowColor{DarkBackground}
\mymulticolumn{2}{x{8.4cm}}{\bf\textcolor{white}{ansible-doc}}  \tn
% Row 0
\SetRowColor{LightBackground}
List all modules & ansible-doc -l \tn 
% Row Count 2 (+ 2)
% Row 1
\SetRowColor{white}
View documentation for a specific module & ansible-doc apt \tn 
% Row Count 4 (+ 2)
% Row 2
\SetRowColor{LightBackground}
View usage examples for a module & ansible-doc -s copy \tn 
% Row Count 6 (+ 2)
\hhline{>{\arrayrulecolor{DarkBackground}}--}
\SetRowColor{LightBackground}
\mymulticolumn{2}{x{8.4cm}}{Shows module documentation.}  \tn 
\hhline{>{\arrayrulecolor{DarkBackground}}--}
\end{tabularx}
\par\addvspace{1.3em}

\begin{tabularx}{8.4cm}{x{5.36 cm} x{2.64 cm} }
\SetRowColor{DarkBackground}
\mymulticolumn{2}{x{8.4cm}}{\bf\textcolor{white}{ansible-config}}  \tn
% Row 0
\SetRowColor{LightBackground}
View current configuration & \seqsplit{ansible-config} dump \tn 
% Row Count 2 (+ 2)
% Row 1
\SetRowColor{white}
View configuration with default values & \seqsplit{ansible-config} list \tn 
% Row Count 4 (+ 2)
% Row 2
\SetRowColor{LightBackground}
View configuration file in use & \seqsplit{ansible-config} view \tn 
% Row Count 6 (+ 2)
\hhline{>{\arrayrulecolor{DarkBackground}}--}
\end{tabularx}
\par\addvspace{1.3em}

\begin{tabularx}{8.4cm}{x{3.76 cm} x{4.24 cm} }
\SetRowColor{DarkBackground}
\mymulticolumn{2}{x{8.4cm}}{\bf\textcolor{white}{ansible-galaxy}}  \tn
% Row 0
\SetRowColor{LightBackground}
Install a role & ansible-galaxy install username.role\_name \tn 
% Row Count 2 (+ 2)
% Row 1
\SetRowColor{white}
Initialize new role structure & ansible-galaxy init my\_new\_role \tn 
% Row Count 4 (+ 2)
% Row 2
\SetRowColor{LightBackground}
List installed roles & ansible-galaxy list \tn 
% Row Count 6 (+ 2)
% Row 3
\SetRowColor{white}
Install roles from requirements.yml & ansible-galaxy install -r requirements.yml \tn 
% Row Count 8 (+ 2)
\hhline{>{\arrayrulecolor{DarkBackground}}--}
\end{tabularx}
\par\addvspace{1.3em}


% That's all folks
\end{multicols*}

\end{document}